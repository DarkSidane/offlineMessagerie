\documentclass{rapport}
\usepackage{lipsum}
\usepackage{tikz} 
\usepackage{listings}
\usepackage{amsfonts} 
\usepackage{listings}
\usepackage[T1]{fontenc}
\usepackage{color}
\usepackage{array,amsfonts}
\usepackage{amsmath}
\usepackage{verbatim}
\usepackage{mathtools,amssymb}
\usepackage{svg} 


\DeclarePairedDelimiterXPP\seq[2]{}{(}{)}{_{#2}}{#1}
\newcommand\myfunc[5]{%
  \begingroup
  \setlength\arraycolsep{0pt}
  #1\colon\begin{array}[t]{c >{{}}c<{{}} c}
             #2 & \to & #3 \\ #4 & \to & #5 
          \end{array}%
  \endgroup}
\definecolor{dkgreen}{rgb}{0,0.6,0}
\definecolor{gray}{rgb}{0.5,0.5,0.5}
\definecolor{mauve}{rgb}{0.58,0,0.82}

\lstset{frame=tb,
  language=Java,
  aboveskip=3mm,
  belowskip=3mm,
  showstringspaces=false,
  columns=flexible,
  basicstyle={\small\ttfamily},
  numbers=none,
  numberstyle=\tiny\color{gray},
  keywordstyle=\color{blue},
  commentstyle=\color{dkgreen},
  stringstyle=\color{mauve},
  breaklines=true,
  breakatwhitespace=true,
  tabsize=3
}
\title{Rapport Programmation orientée objet} %Titre du fichier

\begin{document}

%----------- Informations du rapport ---------

\logo{logos/UCL.png}
\unif{Institut Galilée - Université Paris 13 }
\titre{Service de messagerie instantanée} %Titre du fichier .pdf
\cours{SE} %Nom du cours
\sujet{Système d'exploitation } %Nom du sujet

\enseignant{Kais \textsc{KLAS}} %Nom de l'enseignant
\eleve{Amine \textsc{BOUJEMAOUI}\\
Sidane \textsc{ALP}} 

%----------- Initialisation -------------------
        
\fairemarges %Afficher les marges
\fairepagedegarde %Créer la page de garde
\tabledematieres %Créer la table de matières

%------------ Corps du rapport ----------------
\section{Introduction}
Dans le cadre de la matière Système d'Exploitation, ce rapport présente une modélisation d'un écosystème donné par un texte donné.
\subsection{Énoncé du problème}

L'objectif du projet est de placer l'étudiant dans un cadre applicatif global, dans lequel il devra mettre en situation des concepts vus en cours (création de processus, communication et synchronisation entre les processus).

Il s'agit de mettre en place un service de messagerie instantanée permettant à plusieurs utilisateurs d'une même machine, de se connecter au service simultanément. Un mécanisme d'abonnement permet de gérer les nouveaux utilisateurs et les sortants (abandon du service).

Une fonctionnalité de connexion, permet à tout instant de connaître la liste des connectés, afin de provoquer une discussion avec eux. Cette discussion passe par l'ouverture de fenêtres de dialogues. Elles permettent à deux personnes de discuter entre elles, via ces fenêtres ("chat").

\subsection{Plan}

ooooooooooooooooooooooooooooo

\newpage
\section{Première partie} 

\subsection{Gestion des contacts}

aaaaaaaaaaaaaaaaaaaaaaaaaaaa

\subsection{Fenêtres de dialogue}

bbbbbbbbbbbbbbbbbbbbbbbbbbb

\newpage

\section{Deuxième partie : enregistrement des utilsateurs}

\subsection{Segment de mémoire partagée (SMH)}

cccccccccccccccccccccccccccccccccccccc

\section{Troisième partie : Extension de l'application}

\subsection{Utilisation des sockets}

dddddddddddddddddddddddddddddd

%------------- Commandes utiles ----------------
 \newpage

\section{Conclusion}

eeeeeeeeeeeeeeeeeeeeeeeeeeeeeeeeeeeeeeeeeeeeeeeeeeeeeeeeeee

\subsection{Améliorations possibles}

ffffffffffffffffffffffffffffffffffffffffffffffffffffffffffffffffffffffffffffffffffff

FIN

\end{document}
